
\centerline{\bf \large\MakeUppercase{Lời mở đầu}}
\vspace{20pt}
Ngày nay, tối ưu hóa đã trở thành một lĩnh vực rất phát triển, góp phần quan trọng trong việc ứng dụng khoa học công nghệ vào cuộc sống và sản xuất. Quy hoạch tuyến tính (QHTT) là một lĩnh vực của tối ưu hóa đã được phát triển từ những năm đầu của thế kỷ 20, đến nay toàn bộ lý thuyết toán học cho lĩnh vực này có thể nói là đã rất hoàn thiện. Mặc dù vậy, ứng dụng của QHTT vẫn đóng một vai trò rất quan trọng trong việc giải quyết các bài toán ứng dụng trong cuộc sống và kỹ nghệ. Bài toán QHTT có thể ứng dụng trực tiếp vào các lĩnh vực như sản xuất với mô hình "lập kế hoạch sản xuất", vào giao thông vận tải với mô "bài toán vận tải", vào quản lý con người với mô hình "phân việc"... hoặc nó có thể ứng dụng gián tiếp như những bài toán con trong các phương pháp, các thuật toán giải các bài toán tối ưu phi tuyến, bài toán điều khiển... Đối với các ứng dụng kể trên, bài toán QHTT thường có kích thước lớn, do đó việc xử lý chúng như thông thường là điều không thể. Do đó, việc thiết kế những thuật toán theo hướng giải quyết các bài toán lớn là một trong những vấn đề vẫn đang được quan tâm xử lý hiện nay.

Các bài toán QHTT có kích cỡ trung bình, việc sử dụng phương pháp đơn hình của Dantzig hay các phương pháp điểm trong một cách trực tiếp là rất hiệu quả và tin cậy. Tuy nhiên, qua thực tiễn tính toán và áp dụng, nhiều lớp bài toán kích thước lớn xuất hiện trong nhiều ứng dụng lại có những cấu trúc riêng trên các ràng buộc, đặc biệt là ma trận ràng buộc thường có cấu trúc đường chéo, chéo khối... và thường là những ma trận thưa.

Mục đích của khóa luận này là nhằm tìm hiểu một phương pháp giải quyết các bài toán QHTT kích thước lớn, có cấu trúc. Phương pháp được đề cập ở đây là phương pháp phân rã Dantzig-Wolfe. Trên thực tế, phương pháp phân rã Dantzig-Wolfe là một phương pháp khá tổng quát và được phát triển khá mạnh từ khi xuất hiện không chỉ trong QHTT mà cả trong quy hoạch lồi và các bài toán tối ưu nói chung. Nhìn chung, phương pháp phân rã Dantzig-Wolfe nhằm mục đích phân rã một bài toán QHTT có kích thước lớn thành một số các bài toán có kích thước nhỏ hơn bằng một số phép biến đổi. Sau đó áp dụng phương pháp đơn hình để giải quyết các bài toán phân rã và sử dụng nghiệm của các bài toán phân rã để tạo nên nghiệm cho bài toán ban đầu.

Khóa luận tập trung làm rõ một số vấn đề sau: Trình bày ý tưởng của phương pháp phân rã Dantzig-Wolfe, các khái niệm và tính chất liên quan đến phương pháp, nội dung phương pháp và cuối cùng là các ví dụ cũng như kết quả tính toán bằng số để minh họa cho phương pháp phân rã Dantzig-Wolfe.

Bố cục của khóa luận bao gồm 3 chương và một phụ lục:
\begin{itemize}
\item  Chương 1 của khóa luận trình bày tóm tắt một số kết quả đã biết trong giải tích lồi, thuật toán đơn hình, các định lý và kết quả cơ bản liên quan đến khóa luận. Cuối chương trình bày một số phương pháp lưu trữ ma trận.
\item  Chương 2 của khóa luận tập trung trình bày ý tưởng, các khái niệm và tính chất và nội dung cơ bản của phương pháp phân rã Dantzig-Wolfe giải bài toán quy hoạch tuyến tính lớn có cấu trúc. Cuối chương là một ví dụ bằng số minh họa kết quả tính toán.
\item Chương 3 trình bày quan điểm thực thi của phương pháp phân rã Dantzig-Wolfe.
\item Phụ lục trình bày một số module cơ bản trong lập trình thuật toán.
\end{itemize}
Do thời gian thực hiện khóa luận không nhiều, kiến thức còn hạn chế nên khi làm khóa luận không tránh khỏi những hạn chế và sai sót. Tác giả mong  nhận được sự góp ý và những ý kiến phản biện của quý thầy cô và bạn đọc.
\textrm{Xin chân thành cảm ơn!}
  \begin{flushright}
{\it Hà Nội, ngày 19 tháng 05 năm 2008}

 Sinh viên \hskip 2cm\quad

\vskip 2cm
{\bf Trịnh Văn Hải} \hskip 1cm \quad\ 
 \end{flushright}



